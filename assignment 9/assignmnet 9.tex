% Inbuilt themes in beamer
\documentclass{beamer}

% Theme choice:
\usetheme{CambridgeUS}

% Title page details: 
\title{Assignment 9\\Probability and Random Variables} 
\author{Shreyas Wankhede}
\date{\today}
\institute{IIT Hyderabad}
\logo{\large \LaTeX{}}


\begin{document}

% Title page frame
\begin{frame}
    \titlepage 
\end{frame}

% Remove logo from the next slides
\logo{}


% Outline frame
\begin{frame}{Outline}
    \tableofcontents
\end{frame}


% Lists frame
\section{Question}
\begin{frame}{Question}
\begin{block}{Question 7.10}
We denote by $X_m$ a random variable equal to the number of tosses of a coin until heads shows for the mth time. Show that if\\\vspace{3mm}
 $ P\{h\} = p$ , then $E\{x_m\} = \dfrac{m}{p}. $

\end{block}


\end{frame}


% Blocks frame
\section{Solution}
\begin{frame}{Solution}
As we know,
    \begin{block}{}
        $1+x+x^2+x^3+....x^n+....=\dfrac{1}{1-x}$
    \end{block}
    Differentiating, we obtain,
    \begin{block}{}
        $1+2x+3x^2+4x^3+....nx^{n-1}+...=\sum_{k=1}^{\infty}kx^{k-1}=\dfrac{1}{(1-x)^2}$
    \end{block}
    
\end{frame} 


\begin{frame}{solution}
The Random variable $x_1$ equals the number of tosses until heads show for the first time. Hence $x_1$ takes the values $1, 2, 3, ....$\\with $P(x_1=k)=pq^{k-1}$, Hence,
\begin{block}{from above equation,}
 E$(x_1)=\sum_{k=1}^{\infty}kP(x_1=k)=\sum_{k=1}^{\infty}kpq^{n-1} =\dfrac{p}{(1-q)^2}=\dfrac{1}{p}$
\end{block}

\end{frame}

\begin{frame}
Starting the count after the first head shows,we conclude that the random variable $x_2-x_1$ has the same statistics as the random variable $x_1$.Hence,
\begin{block}{we can write}
E$(x_2-x_1)=E(x_1)$\\\vspace{3mm}
E$(x_2)=2E(x_1)=\dfrac{2}{p}$
\end{block}
 Reasoning similarly, we can conclude that,
 \begin{block}{from all above results}
 E$(x_n-x_{n-1})=E(x_1)$. Hence (induction)\\\vspace{3mm}
 E$(x_n)=E(x_{n-1})+E(x_1)=\dfrac{n-1}{p}+\dfrac{1}{p}=\dfrac{n}{p}$
 \end{block}
\end{frame}

\end{document}